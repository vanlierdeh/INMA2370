\ifx \globalmark \undefined %% This is default.
	\documentclass[twoside,openright,11pt,a4paper]{report}

%\compiler avec xelatex
%\usepackage[applemac]{inputenc}
\usepackage[T1]{fontenc}
\usepackage[utf8]{inputenc} %latin1 est possible
%\usepackage[latin1]{inputenc} %latin1 est possible
\usepackage[francais]{babel}
\usepackage{lettrine}

\usepackage[text={13cm,20cm},centering]{geometry}

\renewcommand{\familydefault}{cmss}

\usepackage{graphicx}
\usepackage{amsmath}
\usepackage{amsfonts}
\usepackage{amssymb}
\usepackage{amsthm}
\usepackage{bm}
\usepackage{color}

\newcommand{\real}{\mathbb{R}}
\newcommand{\mb}{\mathbf}
\newcommand{\bos}{\boldsymbol}

\def \RR {I \! \! R}

\newcommand{\e}{\begin{equation}}  
\newcommand{\ee}{\end{equation}}
\newcommand{\eqn}{\begin{eqnarray}} 
\newcommand{\eeqn}{\end{eqnarray}} 
\newcommand{\eqnn}{\begin{eqnarray*}} 
\newcommand{\eeqnn}{\end{eqnarray*}} 

\newcommand{\bpm}{\begin{pmatrix}}
\newcommand{\epm}{\end{pmatrix}}

%\newcommand{\{\c c}}{\c c}

\newcommand{\bma}{\left(\begin{array}}
\newcommand{\ema}{\end{array}\right)} 
\newcommand{\hh}{\hspace{2mm}}
\newcommand{\hd}{\hspace{5mm}}
\newcommand{\hu}{\hspace{1cm}}
\newcommand{\vv}{\vspace{2mm}}
\newcommand{\vd}{\vspace{5mm}}
\newcommand{\vm}{\vspace{-2mm}}
\newcommand{\teq}{\triangleq}
%\newcommand{\qedb}{\,$\Box$}
\newcommand{\blanc}{$\left. \right.$}
\newcommand{\frts}[2]%
         {\frac{{\textstyle #1}}{{\textstyle #2}}}

\newcommand{\bindex}[3]%
{
\renewcommand{\arraystretch}{0.5}
\begin{array}[t]{c}
#1\\
{\scriptstyle #2}\\
{\scriptstyle #3}
\end{array}
\renewcommand{\arraystretch}{1}
}

\theoremstyle{definition}
\newtheorem{exemple}{{\bf Example}}[chapter]
\newtheorem{theoreme}[exemple]{{\bf Theorem}}
\newtheorem{propriete}[exemple]{{\bf Property}}
\newtheorem{definition}[exemple]{{\bf Definition}}
\newtheorem{remarque}[exemple]{{\bf Remark}}
\newtheorem{remarques}[exemple]{{\bf Remarks}}
\newtheorem{lemme}[exemple]{{\bf Lemma}}
\newtheorem{hypothese}[exemple]{{\bf Hypothesis}}
\newtheorem{exercice}{{\bf Exercise}}[chapter]

\newcommand{\xqedhere}[2]{%
 \rlap{\hbox to#1{\hfil\llap{\ensuremath{#2}}}}}

\newcommand{\xqed}[1]{%
 \leavevmode\unskip\penalty9999 \hbox{}\nobreak\hfill
 \quad\hbox{\ensuremath{#1}}}

\newcommand{\gf}{\fg\,\,}

\newcommand{\cata}[1] %
     {\renewcommand{\arraystretch}{0.5}
     \begin{array}[t]{c} \longrightarrow \\ {#1} \end{array}
     \renewcommand{\arraystretch}{1}}

\usepackage[isu]{caption}
%\usepackage[font=small,format=plain,labelfont=bf,up,textfont=it,up]{caption}
\setlength{\captionmargin}{60pt}

\newcommand{\cqfd}
{%
\mbox{}%
\nolinebreak%
\hfill%
\rule{2mm}{2mm}%
\medbreak%
\par%
}

\pagestyle{headings}

\renewcommand{\sectionmark}[1]{%
\markright{\thesection.\ #1}{}}

\renewcommand{\chaptermark}[1]{%
\markboth{\chaptername\ \thechapter.\ #1}{}}

\makeatletter 
\def\@seccntformat#1{\csname the#1\endcsname.\;} 
\makeatother

\title{ {\Huge {\textbf{Modélisation et analyse  \\ \vspace{4mm} des systèmes dynamiques }}} \\ \vspace{4cm} G. Bastin}

%\title{ {\Huge {\textbf{Modelisation et analyse  \\ \vspace{4mm} des systemes dynamiques }}} \\ \vspace{4cm} G. Bastin}


\date{\today}
	\begin{document} %% Crashes if put after (one of the many mysteries of LaTeX?).
\else 
	\documentclass{standalone}
	\begin{document}
\fi

\graphicspath{ {Chapitre9/images/} }

\setcounter{chapter}{8}
\chapter{Equilibriums stability}
\chaptermark{Stabilité des équilibres}\label{stabeq}



\lettrine[lines=1]{\bf T}{}his chapter is about the stability of equilibriums. Precisely, we are interested in the behaviour of the trajectories of the system in the neighbourhood of the equilibriums.  Let a dynamic system be described by its state model~:
\e
\dot x = f(x,u). \label{sys}
\ee
We assume that the system owns an equilibrium in $(\bar x, \bar u)$. We ask ourselves the two following questions~:\\
\begin{itemize}
\item[a:] If the entry is kept at its equilibrium's value $\bar u$ and if the initial state $x(t_0)$ is in the neighbourhood of the equilibrium's value $\bar x$, how will the trajectories of the system behave ? Under which conditions will the trajectories converge towards $\bar x$?\\
\item[b:] If the entry $u(t)$ is close to $\bar u $ (but not necessarily constant), what can we say about the system's trajectories ? Under which conditions will the trajectories $x(t)$ stay close to $\bar x$?\\
\end{itemize}

\section{Definitions}

\begin{definition} \label{eqstab} {\bf Stable equilibrium}

The equilibrium $(\bar x, \bar u)$ is a {\em stable equilibrium} of the system (\ref{sys}) if 
\eqnn
\forall\epsilon > 0 \;\;\exists \delta>0 \mbox{ s.t.} \| x(t_0)-\bar x\| < \delta
\Rightarrow \| x(t, x(t_0), \bar u) -\bar x \| < \epsilon \;\; \forall t \geq
t_0.
\eeqnn
If this condition is not satisfied, the equilibrium is {\em unstable}. \qed
\end{definition}
This definition is to be interpreted as follows.We want to characterize how the trajectory $x(t)$ stays close to the equilibrium point $\bar x$ for every $t \geq t_0$ when the entry is constant $(u(t) = \bar u \; \forall t \geq t_0$). For that, we measure the proximity with the norm $\| \;\;  \|$ and we impose that the solutions $x(t)$ stay within the region bound by $ \| x(t) -\bar x \| < \epsilon$, meaning in a "tube" of radius $\epsilon$ around the trajectory $x(t) = \bar x$. If this goal is achievable for an initial condition $x(t_0)$ close to the equilibrium (meaning that $ \| x(t_0) -\bar x \| < \delta$), then we say that the equilibrium is stable.  The equilibrium is said unstable otherwise.

The previous definition is the weakest form of stability considered in this chapter. In particular, it does not imply that the trajectories $x(t)$ converge to the equilibrium point.

\begin{definition} \label{eqatt} {\bf Attractive equilibrium}

The equilibrium $(\bar x, \bar u)$ is an {\em attractive equilibrium} of 
(\ref{sys}) if
\eqnn
\exists \delta>0 \mbox{ s.t.} \| x(t_0)-\bar x \| < \delta
\Rightarrow  \lim_{t \rightarrow \infty} \|x(t, x(t_0), \bar u) -\bar x \|=0. \qed
\eeqnn
\end{definition}
An attractive equilibrium $\bar x$ is therefore a point to which the solutions $x(t)$ converge if they start close enough of $\bar x$. It has to be noted that stability and attractiveness are two different notions and that they don't imply each other. 
\begin{definition} \label{eqast} {\bf Asymptotically stable equilibrium}

The equilibrium $(\bar x, \bar u)$ is an {\em asymptotically stable equilibrium} if it is stable and attractive.
\qed
\end{definition}
A set of initial states $x(t_0)$ from which the trajectories converge to an asymptotically stable equilibrium is called {\em basin of attraction}.
The asymptotic stability is the property that is usually sought for in practice.  However, we have to notice that this definition above does not indicate the speed at which the trajectory $x(t)$ converge towards the equilibrium. That's why the notion of exponential stability was introduced, which allows to characterize that speed.
\begin{definition} \label{expstab} {\bf Exponential stability}

The equilibrium $(\bar x, \bar u)$ is an {\em exponentially stable equilibrium} if
\eqnn
&&\forall\epsilon > 0 \;\;\exists \; a>0, b>0 \mbox{ and } \delta>0 \text{ s.t.} \\ 
&& \| x(t_0)-\bar x \| < \delta 
\Rightarrow \| x(t, x(t_0), \bar u) -\bar x \| \leq a \| x(t_0)-\bar x \|
e^{-bt}\;\; \forall t \geq t_0. \qed
\eeqnn
\end{definition}
It is obvious that the exponential stability implies the asymptotic stability but the opposite is not necessarily true.

\section{Lyapunov's first method (indirect method)}

Lyapunov's first method is based on the examination of the system's linearization $\dot x = f(x,\bar u)$ around the equilibrium $(\bar x, \bar u)$. Precisely, we study the eigenvalues $\lambda_i (A)$ of the Jacobian matrix estimated at the equilibrium~:
\eqnn
A = \dfrac {\partial f}{\partial x} (\bar x, \bar u).
\eeqnn
Under this method, the stability properties of $(\bar x, \bar u)$ are expressed as follows.

\begin{theoreme}\label{premLyap}{\bf Lyapunov's first method.}

\begin{enumerate}
\item If all the eigenvalues of the Jacobian matrix have a real part strictly negative ($\forall i,\;Re(\lambda_i(A))<0$), the equilibrium $(\bar x, \bar u)$ is exponentially stable.
\item If the Jacobian matrix owns at least one eigenvalue with a real part strictly positive ($\exists i, \;Re(\lambda_i(A)) > 0$), the equilibrium $(\bar x, \bar u)$ is unstable.
\qed
\end{enumerate}
\end{theoreme}
This theorem does not allow to say if the equilibrium is stable or unstable if the Jacobian matrix owns at least one eigenvalue with a real part equal to zero and no eigenvalue with a eigenvalue with a strictly positive real part. In this case, the system's trajectories converge to a sub-set (a variety) whose dimension is the number of eigenvalues of the Jacobian equal to zero and the equilibrium's stability can be examined in this sub-set with the second method.

\section{Lyapunov's second method (direct method)}

As we juste saw, Lyapunov's first method is simple to apply but it only allows to study the equilibrium's stability partially. Furthermore, it does not give any indication on the size of the basins of attraction.  The second method is more difficult to implement but, in return, is more general in scope.  It is based on the definition of a particular function noted $V(x)$ and called {\em Lyapunov's function}, which is decreasing along the system's trajectories within the basin of attraction. Before giving the different stability theorems, we start with an example.

\begin{exemple}\label{exbras}{\bf A robot's arm with one degree of freedom.}

We consider a robot's arm with one degree of freedom, with linear viscous friction and under a constant moment (see following figure). 
The state model of the the system is the following (see chapter 2 and 8, section 8.3.1)~:
\begin{equation*} \begin{split}
\dot x_1 &=x_2, \\
\dot x_2 &= J^{-1}(- mgb \sin x_1 - k x_2 + \bar u).
\end{split} \end{equation*}
In these equations, $x_1$ is the angular position, $x_2$ the angular speed, $J$ the inertia, $m$ the mass, $b$ the distance between the anchor point and the center of mass, $k$ the friction coefficient and $\bar u$ the constant moment applied to the robot's arm.

Let's consider the case where $0 < \bar u < mgb$. The system owns two equilibriums verifying the two following relations~:
\eqnn
\bar x_1 = \arcsin (\dfrac {\bar u}{mgb}), \hh
\bar x_2 = 0.
\eeqnn
In accordance with the physical intuition and as we can verify by examining the Jacobian matrix's eigenvalues, there is an asymptotically stable equilibrium in the "lower" position and an unstable equilibrium in the "higher" position (see figure \ref{bras}).
\begin{figure}[h]
\begin{center}
\includegraphics[width=6cm]{bras}
\caption{The robot's arm in a vertical plane}
\label{bras}
\end{center}
\end{figure}

Let's consider the following function~:  
\eqnn
V(x_1,x_2) = \dfrac {J}{2} x_2^2 +  mgb(1 - \cos x_1) -\bar u x_1. 
\eeqnn
This function has the dimension of an energy. Indeed, the first term ($J x_2^2/2$) is the kinetic energy, the second one ($mgb(1-\cos x_1)$) is the potential energy and the third one ($\bar u x_1$) is the energy spend by the moment $\bar u$ pour elevate the arm to the angular position $x_1$ . The equilibrium in the "lower" position belongs to the domain
\eqnn
D = \{ (x_1,x_2) : - \pi /2 < x_1 < \pi /2, -a < x_2 < a \}
\eeqnn
(a is any real positive number).  In this domain, the function $V(x_1,x_2)$ is a function that satisfies the following conditions~:
\begin{itemize}
\item [(i)] $V(x_1,x_2) : D \rightarrow R$ is continually differentiable.
\item [(ii)] $V(x_1,x_2) > V(\bar x_1, \bar x_2)$ for all $(x_1,x_2) \neq (\bar x_1, \bar x_2)$ in $D$ (meaning $V$ is minimum at equilibrium).
\item [(iii)] $\dot V(x_1,x_2) \leq 0$ outside the equilibriumen in $D$ because
\eqnn
\dot V(x_1,x_2) &=& \dfrac {\partial V}{\partial x_1} \dot x_1+  \dfrac {\partial V}{\partial x_2} \dot x_2 \\
&=& [mgb \sin x_1 - \bar u][x_2] + [J x_2][ J^{-1}(- mgb \sin x_1 - k x_2 + \bar u]\\
&=& - k x_2^2.
\eeqnn
\end{itemize}
Under these conditions, as we will see with the next theorem, there is a bound neighbourhood of the equilibrium in which $V(x_1,x_2)$ decreases along the system's trajectories as long as the speed $x_2 \neq 0$ and moves towards the minimum of $V$ that corresponds to the equilibrium. We can see that the equilibrium is stable according the definition \ref{eqstab}. However, we notice that $V(x_1,x_2)$ ceases to decrease if $x_2=0$ (speed equal to zero). Could we have a speed identically equal to zero elsewhere than the equilibrium ? In fact we can't, because a speed identically equal to zero implies an acceleration identically equal to zero, which implies that the system is at an equilibrium.
\qed
\end{exemple}
This example illustrates the key part of Lyapunov's second method which is as follows. 

\begin{theoreme}\label{stabLyap}{\bf \og Lyapunov's \fg stability.}

The equilibrium $(\bar x, \bar u)$ of the system $\dot x = f(x,\bar u)$ is stable if there is a function $V(x): D \rightarrow R$ continually differentiable with the following properties~:\\
\begin{itemize}
\item[(i)] $D$ is an open set of $R^n$ and $\bar x \in D$;\\
\item[(ii)] $V(x) > V(\bar x)$ $\forall x \neq \bar x$ in $D$ ($V(x)$ is minimum at $\bar x$);\\
\item[(iii)] $\dot V(x) \leq 0$ $\forall x \neq \bar x$ in $D$. \qed
\end{itemize}
\end{theoreme}
In other words, this theorem means that a sufficient condition to have the stability of the equilibrium $(\bar x, \bar u)$ is to have a positive-definite function $V(x) - V(\bar x)$ whose temporal derivative $\dot V(x)$ is negative semi-definite in a neighbourhood of $\bar x$. The temporal derivative $\dot V(x)$ is calculated as follows~:
\eqnn
\dot V(x) = \dfrac {dV}{dt} = \dfrac {\partial V}{\partial x} \dot x = \dfrac {\partial V}{\partial x} f(x, \bar u) = \sum_{i=1}^n \dfrac {\partial V}{\partial x_i} f_i(x, \bar u).
\eeqnn
The conditions (ii) and (iii) of the theorem \ref{stabLyap} imply that, for a constant $c$ close enough to $V(\bar x)$, the set~:
\eqnn
\Omega_c = \{ x \in D : V(\bar x) \leq V(x) \leq c \}
\eeqnn
is a compact (meaning a closed and bound set) invariant. To demonstrate it, 
let's choose a positive constant $r$ such that the closed ball
\eqnn
B_r = \{ x \in R^n :  \| x - \bar x \| \leq r \}
\eeqnn
is continuous in $D$. Let's define~:
\eqnn
\alpha = \min_{\| x \| = r} V(x).
\eeqnn
We only have to choose any constant $c$ in the open interval $(0,\alpha)$ to define a set $\Omega_c$ which is included in $B_r$ and therefore compact. Let's now assume that $x(t_0) \in \Omega_c$. Then, by the condition (iii), we have~:
\eqnn
\dot V(x) \leq 0 \;\;\; \Rightarrow \;\;\; V(\bar x) \leq V(x(t)) \leq V(x(t_0)) \leq c \;\;\; \forall t,
\eeqnn
which shows that $\Omega_c$ is invariant.

The example \ref{exbras} is an application of this theorem that shows that the robot's arm's equilibrium est stable. En reality, we know intuitively that the equilibrium is {\em asymptotically} stable (meaning stable and attractive). One way to demonstrate that an equilibrium is asymptotically stable is to have a Lyapunov function whose temporal derivative $ \dot V(x)$ is strictly negative-defined (and not only negative semi-defined as in the example 9.6). Indeed, in this case, the Lyapunov function strictly decreases along the system's trajectories, until it reaches (asymptotically) the minimum corresponding exactly to the equilibrium.

\begin{theoreme}\label{stabasymp}{\bf Asymptotical stability.}

The equilibrium $(\bar x, \bar u)$ of the system $\dot x = f(x,\bar u)$ is asymptotically stable if there is a function $V(x): D \rightarrow R$ continuously differentiable with the following properties~:\\
\begin{itemize}
\item[(i)] $D$ is an open set $R^n$ and $\bar x \in D$;\\
\item[(ii)] $V(x) > V(\bar x)$ $\forall x \neq \bar x$ in $D$ ($V(x)$ is minimum at $\bar x$);\\
\item[(iii)] $\dot V(x) < 0$ $\forall x \neq \bar x$ in $D$. \qed
\end{itemize}
\end{theoreme}

As we have seen in the robot's arm example, we usually find a Lyapunov function whose derivative is only negative semi-definite, which does not allow to conclude that the system is stable, using the previous theorem.  The difficulty comes notably from the fact that, by analyzing the function $\dot V(x)$, we don't use the fact that the different state variables $x_i$ are not independent but are linked by the equations of the system's dynamic. LaSalle studied this matter in detail and has come with a invariance principle which allows to analyze the asymptotic stability of the equilibriums in the case of a negative semi-definite function $\dot V(x)$.
\newpage
\begin{theoreme}\label{lasalle}{\bf Lasalle invariance principle.}

The equilibrium $(\bar x, \bar u)$ of the system $\dot x = f(x,\bar u)$ is asymptotically stable if there is a function $V(x): D \rightarrow R$ continuously differentiable with the following properties~:\\
\begin{itemize}
\item[(i)] $D$ is an open set of $R^n$ and $\bar x \in D$;\\
\item[(ii)] $V(x) > V(\bar x)$ $\forall x \neq \bar x$ in $D$ ($V(x)$ is minimum at $\bar x$);\\
\item[(iii)] $\dot V(x) \leq 0$ $\forall x$ in  $D$;\\
\item[(iv)] the set $S \subset D$ such that $\dot V(x) = 0$ does not contain any {\em trajectory} of the system other than $x(t) = \bar x$. \qed
\end{itemize}
\end{theoreme}

\begin{exemple}{\bf  Un bras de robot à un degré de liberté (suite).}

Let's consider again the robot's arm model (see Example \ref{exbras})
\begin{equation} \begin{split} \label{bras}
\dot x_1 &= x_2, \\
\dot x_2 &= J^{-1}(- mgb \sin x_1 - k x_2 + \bar u), 
\end{split} \end{equation}
for which the energy function 
\eqnn
V(x_1,x_2) = \dfrac {J}{2} x_2^2 +  mgb(1 - \cos x_1) -\bar u x_1 
\eeqnn
is a Lyapunov function. Among the system's trajectories, the Lyapunov function's derivative is negative semi-definite~:
\eqnn
\dot V(x_1,x_2)= - k x_2^2 \leq 0.
\eeqnn
The set $S \subset D$ such that $\dot V(x) = 0$ is therefore the following~:
\eqnn
S = \{ (x_1,x_2) \in D : x_2 = 0 \}.
\eeqnn
We can observe that every trajectory within $S$ is such that the speed $x_2(t)$ is identically equal to zero (which we note $x_2(t) \equiv 0$). This immediately implies that $\dot x_2(t) \equiv 0$. By equation (\ref{bras}), this implies that $mgb \sin x_1(t) - \bar u \equiv 0$. And therefore that the only trajectory within $S$ is indeed the equilibrium's trajectory. Thus the equilibrium is asymptotically stable.  The reasoning that we just did is formalized as follows~:
\eqnn
\text{trajectory }(x_1(t),x_2(t)) \in S &\Rightarrow& x_2(t) \equiv 0 \\
&\Rightarrow& \dot x_2(t) \equiv 0 \\
&\Rightarrow& mgb \sin x_1(t) - \bar u \equiv 0 \\
&\Rightarrow& x_1(t) = \bar x_1, x_2(t) = \bar x_2.
\eeqnn
\qed
\end{exemple}

\section{Basin of attraction and global convergence}

In the demonstration of the theorem \ref{stabLyap}, we have seen that the domain $\Omega_c$ is an invariant of the system. If the equilibrium is asymptotically stable, it means that every trajectory that starts at any point of $\Omega_c$ converges to the equilibrium. That is the reason why this set is called "basin of attraction". Lyapunov second method allows us to characterize the size of the basin of attraction, information that we could not get with the first method.  That's why it can be interesting to look for a Lyapunov function, even if the equilibrium's stability can be easily demonstrated with the linearization.

One particularly interesting case is when the equilibrium point is unique and the basin of attraction contains the whole state space. That case is referred as {\it global asymptotic stability} whose existence conditions are developed in the following theorem.

\begin{theoreme}\label{stabglob}{\bf Global asymptotic stability.}

The equilibrium $(\bar x, \bar u)$ of the system $\dot x = f(x,\bar u)$ is globally asymptotically stable if it is asymptotically stable and if~:\\
\begin{itemize}
\item[(i)] $D$ = $\RR^n$;\\
\item[(ii)] $|x| \rightarrow \infty \Rightarrow |V(x)| \rightarrow \infty$. \qed
\end{itemize}
\end{theoreme}

\section{The energy as a Lyapunov function}

The choice of a Lyapunov function that is appropriate for the analysis of the equilibriums stability of a dynamic system is usually quite difficult.  As we have shown with the robot with one degree of liberty example, the energy can be a good start for several physical systems.  Let's examine that on a few examples.\\

\noindent {\bf Mechanical systems}

The general equation of the dynamics of a mechanical system is (cfr Chapter 2) 
$$
M(q) \ddot q + C(q, \dot q) \dot q + g(q) +k(q) +h(\dot q) =G \bar u
$$
Here we consider the particular case where the cinematic matrix $G$ is constant.
Let's take as function $V(q, \dot q)$ the following function :
$$
V(q,\dot q) = \frac{1}{2} \dot q^T M(q) \dot q + E_{p}(q) - q^TG \bar u
$$
The first therm is the kinetic energy, the second one is the potential energy and the third one is the work realized by the forces and moments applied.  The derivative of this function along the trajectories is calculated as follows~:
\begin{equation*} \begin{split}
\dot V &=  \dfrac{1}{2} \dot q^T[\dot M(q) -2C(q,\dot q)]\dot q -
\dot q^Th(\dot q)\\
&= -\dot q^Th(\dot q),
\end{split} \end{equation*}
(because the matrix $\dot M(q) -2C(q,\dot q)$ is antisymmetric, see chapter 2). This value is well negative semi-definite for reasonable choice of viscous friction models.\\

\noindent {\bf Electrical circuits}

Let's take for example the non-linear RLC circuit from Chapter 8 (sec. 8.3.1). The state equations are :
\begin{equation*} \begin{split}
L \dot x_1 &= - r(x_1) - x_2 + \bar u, \\
C \dot x_2 &= x_1.
\end{split} \end{equation*}
In these equations, $x_1$ denotes the current and $x_2$ the tension. $L$ and $C$ are the inductance et capacitor of the circuit while $r(x_1)$ is a non-linear resistor. We assume that the function $r(x_1)$ is monotonously increasing and passes through the origin $r(0)=0$.

Let's take for Lyapunov function, the following function which has the dimension of an energy~:
 $$
 V(x_1, x_2) = \dfrac{1}{2}Lx_1^2+\dfrac{1}{2}C(x_2 -\bar u)^2 \geq 0
 $$
 This function is positive and minimum at the equilibrium $(\bar x_1, \bar x_2) =(0, \bar u)$~: $V(\bar x_1, \bar x_2) = 0$.
On the other hand, we have~:
\begin{equation*} \begin{split}
 \dot V &=  \dfrac{\partial V}{\partial x_1} \dot x_1 +
 \dfrac{\partial V}{\partial x_2} \dot x_2\\
 &= -x_1r(x_1) \leq 0.
\end{split} \end{equation*}
The equilibrium is therefore stable and, using the Lasalle invariant principle, we can even conclude the asymptotic stability.\\

\section{Linear systems}

\noindent Let a system $\dot x = Ax + B\bar u$ and an equilibrium $(\bar x, \bar u)$. We define as Lyapuno function $V(x) = (x-\bar x)^T P(x-\bar x)$ where
$P$ is a symmetric positive-definite matrix.
\begin{equation*} \begin{split}
\dot V(x) &= \dot x^TP(x-\bar x) +(x-\bar x)^TP\dot x \\
&= (\bar u^TB^T+x^TA^T)P(x-\bar x) +(x-\bar x)^TP(Ax +B\bar u)\\
&= (x-\bar x)^TA^TP(x-\bar x) +(x-\bar x)^TP A(x-\bar x) \\
&= -(x-\bar x)^TQ(x-\bar x),
\end{split} \end{equation*}

\begin{equation} 
\hspace{-5.4cm} \text{with} -Q = A^TP+PA. \label{SL}
\end{equation}
This last equation is called \og Lyapunov matrix equation \fg.  If
it has a solution $Q$ that is positive-definite, then the function $V$ will be a Lyapunov function for the system.  The reasoning can also be reversed : we take a matrix $Q$ positive-definite and we use the following theorem to conclude that the function $V$ is indeed a Lyapunov function for the system et that the equilibrium is asymptotically stable. 

\begin{theoreme}  Let a $n$-order real matrix $A$.  
For every matrix $Q$ positive-definite, (\ref{SL}) owns a unique solution $P$ positive-definite if and only if $A$ is a Hurwitz matrix (all its eigenvalues have a strictly negative real part).
\qed
\end{theoreme}

\section{\og Bound entry - Bound state \fg stability}
It would often be illusory to be able to apply a perfectly constant entry to a real dynamic system.  In practice, because of various sources of perturbation and uncertainty, the entry will usually be a signal $u(t)$ slightly varying in the neighbourhood of the wanted equilibrium value. It is therefore relevant to take interest in the evolution of the system's state when $u(t)$ is a bound signal close to $\bar u$. We begin with the case of a linear system
\e \label{systlin}
\dot x = Ax+Bu.
\ee
For an given initial condition $x(t_0) = x_0$ and a given entry $u(t)$, the system's trajectory is explicitly given as
\eqnn
x(t) = e^{A(t-t_0)}x_0 + \int_{t_0}^t e^{A(t-\tau)} B u(\tau) d \tau.
\eeqnn
Let's consider the equilibrium $(\bar x = 0, \bar u = 0)$. This equilibrium is asymptotically stable if and only if the matrix $A$ is a Hurwitz matrix. In that case, $\| e^{At} \|$ is bound for all $t$ and there is positive constants $k$ and $\lambda$ such that
\eqnn
\| e^{A(t-t_0)} \| \leq ke^{-\lambda (t-t_0)}.
\eeqnn
It can be inferred that
\begin{align}
\| x(t) \| &\leq ke^{-\lambda (t-t_0)}\| x_0 \| + \int_{t_0}^t e^{-\lambda(t-\tau)} \| B \| \| u (\tau) \| d \tau \nonumber \\
&\leq ke^{-\lambda (t-t_0)}\| x_0 \| + \dfrac{k \| B \|}{\lambda} \sup_{t_0 \leq \tau \leq t} \| u(\tau) \|. \label{zz}
\end{align}
We can immediately see that a bound entry $u(t)$, whatever its amplitude, does generate a bound state $x(t)$. We can also see that the effect of the initial condition $x_0$ fades away with time and that the \og ultimate boundary \fg \, of $x(t)$ is simply proportional to the boundary of $u(t)$ :
\eqnn
\limsup_{t \rightarrow +\infty} \| x(t) \| \leq \dfrac{k \| B \|}{\lambda} \| u \|_{{\cal L}_\infty}.
\eeqnn
Let's now see how these results extend to non-linear systems.  We consider the system
\eqn \label{systnl}
\dot x = f(x,u) \text{ with the equilibrium } (\bar x, \bar u)
\eeqn
and we assume that the function $f(x,u)$ is continuously differentiable in a neighbourhood of the equilibrium.

\begin{theoreme}{\bf (local) EBEB stability}

If the equilibrium $(\bar x, \bar u)$ of the system (\ref{systnl}) is asymptotically stable, 
\itemize
\item[(i)] there are three positive constants $c_1$, $c_2$ et $c_3$ such that, for all initial state $x_0$ with $\|x_0 - \bar x\| < c_1$ and all entry signal $u(t)$ with $\| u(t) - \bar u \| < c_2$, the solution $x(t)$ is bound : $\| x(t) - x_0 \| < c_3$ $\forall t\geq t_0$;
\item[(ii)] there is a positive constant $c_0$ and a continuous function $\alpha : [0,a) \rightarrow [0, +\infty)$ passing through the origin (meaning $\alpha(0) = 0$) and increasing\footnote{such a function is called a ${\cal K}$ class function} such that, for all entry signal $u(t)$ with $\| u(t) - \bar u \| < c_0$ $\forall t\geq t_0$, the \og ultimate boundary \fg \, of $x(t)$ is an increasing function of the boundary of $u(t)$ :
\eqnn
\limsup_{t \rightarrow +\infty} \| x(t) \| \leq \alpha( \| u \|_{{\cal L}_\infty}). \qed
\eeqnn
\end{theoreme}

In the case where the system (\ref{systnl}) is globally defined and owns an unique equilibrium, we also have the following global property.

\begin{theoreme}{\bf (global) EBEB stability}

If the function $f(x,u)$ is globally continuously differentiable and is globally a Lipschitz function in $(x,u)$, if the equilibrium $(\bar x, \bar u)$ is globally exponentially stable, then 
\itemize
\item[(i)] for all initial condition $x_0$ and all entry signal $u(t)$, the solution $x(t)$ is bound;
\item[(ii)] the \og ultimate boundary \fg \, of $x(t)$ is an increasing function of the boundary of $u(t)$.
\qed
\end{theoreme}
We have to notice that this last theorem is quite restrictive.  There are in fact numerous dynamic systems for which the function $f(x,u)$ is not globally a Lipschitz function and yet own a global EBEB property.
However, the exponential stability condition for the equilibrium is crucial.  Indeed, if the equilibrium is globally asymptotically stable but not globally exponentially stable, then the system (\ref{systnl}) is not necessarily EBEB stable, even if $f(x,u)$ is globally a Lipschitz function.

\section{Exercices}

\begin{exercice}{\bf Un réacteur chimique}

Dans un réacteur continu se déroulent deux réactions chimiques en phase liquide à volume constant faisant intervenir trois espèces chimiques X$_1$, X$_2$, X$_3$. La dynamique du réacteur est décrite par le modèle d'état suivant ($x_i$ désigne la concentration de X$_i$):
\begin{align*}
\dot x_1 &= -x_1^2x_2 - dx_1 + du, \\
\dot x_2 &= x_1^2x_2 - (d+k)x_2, \\
\dot x_3 &= 2kx_2 - dx_3.
\end{align*}
\begin{enumerate}
\item De quelles réactions s'agit-il ? (loi d'action des masses). 
\item Dans l'orthant positif ($x_1 \geq 0$, $x_2 \geq 0$, $x_3 \geq 0$) déterminer le ou les équilibres pour  $\bar u > 0$. Indiquer combien il y a d'équilibres pour chaque valeur de $\bar u$ et expliciter les conditions d'existence. 
\item Analyser la stabilité des équilibres par la première méthode de Lyapunov.
\end{enumerate}
\end{exercice}
\vv

\begin{exercice}{\bf Circuit RLC}

Soit le circuit RLC parallèle illustré ci-dessous :
\begin{figure}[h]
\begin{center}
\includegraphics[width=7cm]{RLC1}
\end{center}
\end{figure}

\noindent La caractéristique courant-tension $i=g(v)$ de la résistance non-linéaire est une fonction monotone croissante telle que représentée sur la figure suivante~:
\begin{figure}[h]
\begin{center}
\includegraphics[width=4.8cm]{RLC2}
\end{center}
\end{figure}
\begin{enumerate}
\item Etablir un modèle d'état du système.
\item Calculer les équilibres.
\item En utilisant l'énergie comme fonction de Lyapunov, analyser la stabilité globale des équilibres par la seconde méthode de Lyapunov.
\end{enumerate}
\end{exercice}
\vv

\begin{exercice}{\bf Réacteur chimique}

Soit un réacteur chimique de type CSTR où se produit la réaction suivante
:
$$
X_1 + X_2 \longrightarrow 2X_2.
$$

\begin{enumerate}
\item Etablir un modèle d'état sous les hypothèses de modélisation suivantes
:\\
- cinétique décrite par la loi d'action des masses\\
- réacteur alimenté en $X_1$ uniquement ($x_{1,in} =$ cste)\\
- le débit volumique d'alimentation est l'entrée.
\item Montrer que ce système possède un équilibre dans l'orthant positif.
\item Montrer que $V = x_1-\bar x_1 \ln x_1+x_2-\bar x_2 \ln x_2$ est une
fonction de Lyapunov dans l'orthant positif.
\item Démontrer que l'équilibre est globalement asymptotiquement stable dans
l'orthant positif.
\end{enumerate}
\end{exercice}
\vv

\begin{exercice}{\bf Système mécanique}

On considère un système mécanique à un degré de liberté.  La variable de
position est $x_1$.  Ce système est soumis à une force dérivant d'un potentiel
et à un frottement visqueux linéaire.

L'énergie potentielle est donnée par $$E_p(x_1) = \int^{x_1}_0 \dfrac{\sigma}{K +
|\sigma|} d \sigma.$$\\

\begin{enumerate}
\item Etablir un modèle d'état du système.
\item Calculer les équilibres.
\item Analyser la stabilité des équilibres par la méthode directe de
Lyapunov.
\end{enumerate}
\end{exercice}
\vv

\begin{exercice}{\bf Modélisation d'un neurone}

Le modèle de Naka-Rushton décrivant la dynamique d'un neurone dans la mémoire à court terme est donné par les équations d'état suivantes~:
\eqnn
\dot x_1 = - x_1 + \frac{u x_2}{1+x_2}, \\
\dot x_2 = - x_2 + \frac{u x_1}{1+x_1}.
\eeqnn
\begin{enumerate}
\item Montrer que le système est positif
\item  Analyser l'existence et la stabilité des équilibres dans l'orthant positif (première méthode de Lyapunov).
\item Pour $u$ constant, $0 < \bar u < 1$, montrer que toutes les trajectoires dans l'orthant positif convergent vers l'origine à l'aide de la fonction de Lyapunov $V = (1/2) (x_1^2 + x_2^2)$.
\end{enumerate}
\end{exercice}
\vv

\begin{exercice} Soit le système dynamique :
\eqnn
\dot x_1 &=& -x_1+x_2,\\
\dot x_2 &=& -x^3_1(1+u^2).
\eeqnn
Avec une fonction de Lyapunov de la forme
$$V = ax^\alpha_1 + bx^\beta_2$$
où $a, b, \alpha, \beta$ sont des constantes positives à
déterminer, montrer que, pour toute entrée constante $u(t) = \bar u$, ce système possède un équilibre unique et
globalement asymptotiquement stable.
\end{exercice}
\vv

\begin{exercice}
Soit le systèmes dynamique :
\begin{align*}
\dot x_1 &= - \phi(x_1) + \phi(x_2), \\
\dot x_2 &= \phi(x_1) - 2 \phi(x_2) + u.
\end{align*}
La fonction $\phi(x) : \RR \longrightarrow \RR$ possède les propriétés suivantes :\\
\begin{itemize}
\item[a)] $\phi(x)$ est une bijection,\\
\item[b)] $\phi(x)$ est $C^\infty$,\\
\item[c)] $\phi(x)$ est strictement monotone croissante ($d\phi$/$dx$ $> 0$, $\forall x \in \RR$),\\
\item[d)] $\phi(x)$ passe par l'origine ($\phi(0) = 0$).
\end{itemize}
\vspace{8mm}
\noindent 1. Démontrer qu'il s'agit d'un système à compartiments. \\

\noindent 2. Pour une entrée constante strictement positive ($\bar u > 0$), expliciter les conditions sous lesquelles le système possède un équilibre unique dans l'orthant positif.\\

\noindent 3. Démontrer que cet équilibre, s'il existe, est globalement asymptotiquement stable à l'aide de la fonction de Lyapunov
\begin{equation*}
V(x_1,x_2) = \int_{\bar x_1}^{x_1} (\phi(s) - \bar u) ds + \int_{\bar x_2}^{x_2} (\phi(s) - \bar u) ds.
\end{equation*}

\end{exercice}

\end{document}
